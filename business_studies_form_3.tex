\documentclass[12pt,a4paper, openany]{book}

% =========================
% PACKAGES
% =========================
\usepackage[utf8]{inputenc}
\usepackage[T1]{fontenc}
\usepackage{lipsum}
\usepackage{graphicx}
\usepackage{hyperref}
\usepackage{amsmath, amssymb}
% =========================
% For APA referencing
\usepackage[style=apa,backend=biber]{biblatex}
\DeclareLanguageMapping{english}{english-apa}

% Add your .bib file
\addbibresource{business_studies_form_3.bib}
% =========================
% BEGIN DOCUMENT
% =========================
\begin{document}

% ---- FRONT MATTER ----
\frontmatter
\title{Business Studies \\ form 3}
\author{Sir Brown}
\date{\today}

\maketitle

\chapter*{Dedication}
To my students at Kamuzu Barracks CDSS.

\chapter*{Preface}
This book provides clear and easy-to-understand notes for Business Studies.
The motivation for writing it arose from the observation that many of my students
do not have access to quality Business Studies materials, and that most existing
books often contain content that is even challenging for undergraduate students.
To make high school Business Studies more accessible and understandable,
I created this book.

\tableofcontents

% ---- MAIN MATTER ----
\mainmatter
\chapter{Foreign Trade}

\section{Introduction}
Trade is defined as the exchange of \textbf{goods} and \textbf{services} with the main aim of making \textbf{profits}.
The exchange of goods and services may take place within a country or across countries.
This gives rise to two types of trade, namely:
\begin{enumerate}
	\item \textbf{Domestic Trade}
	\item \textbf{Foreign Trade}
\end{enumerate}

\section{Domestic Trade}
Domestic Trade is the exchange of goods and services between individuals or organisations located within the same country.
It is mostly characterised by the use of the \textbf{local currency}.
This type of trade is also called \textbf{Home Trade} or \textbf{Local Trade}.

\section{Foreign Trade}

\subsection{Meaning of Foreign Trade}
Foreign Trade is the exchange of capital, goods, and services across the boundaries of a country.
It may occur between governments, individuals, or organisations in different countries.
For example, Malawi sells its tobacco to Japan and America, and imports cars and cellphones from China and India.
Since Foreign Trade takes place between countries that use different currencies, transactions are usually conducted using a common currency such as the US Dollar (\$) or South African Rand (R).

\subsection{Basis of Foreign Trade}
Different countries have different resource \textbf{endowments}.
For example, China can produce cars and cellphones, which Malawi cannot, while Malawi produces tobacco, which China cannot produce in sufficient quantities.
These differences in resources and capabilities give rise to Foreign Trade by allowing countries to buy goods and services they cannot produce themselves, and sell surplus production to other countries.
No country can produce all the goods and services required by its citizens.
Countries \textbf{specialise} in producing certain commodities and rely on Foreign Trade to acquire other commodities produced by other countries.

\vspace{0.5cm}

It is important to note that Foreign Trade may occur between two countries, which is called \textbf{Bilateral Trade}.
It may also involve more than two countries, in which case it is referred to as \textbf{Multilateral Trade}.

\subsection{Divisions of Foreign Trade}
Foreign trade can be divided into three main categories depending on the nature of transactions:

\begin{enumerate}
	\item \textbf{Import Trade}:
	      This is the purchase of goods and services from foreign countries for use in the domestic market.
	      Example: Malawi imports \textit{cars} from Japan and \textit{cellphones} from China.

	\item \textbf{Export Trade}:
	      This is the sale of domestic goods and services to foreign countries.
	      Example: Malawi exports \textit{tobacco} and \textit{tea} to other countries.

	\item \textbf{Entrepot Trade}:
	      This is the re-export of goods. A country imports goods not for its own use, but for the purpose of exporting them to another country.
	      Example: Singapore imports crude oil, refines it, and then re-exports petroleum products to other countries.
\end{enumerate}

\subsection{Major Imports and Exports of Malawi}
The following are some of Malawi’s major imports and exports:

\textbf{Major Exports:}
\begin{itemize}
	\item Tobacco
	\item Tea
	\item Sugar
	\item Coffee
	\item Groundnuts
\end{itemize}

\textbf{Major Imports:}
\begin{itemize}
	\item Motor vehicles and spare parts
	\item Petroleum and petroleum products
	\item Machinery and equipment
	\item Fertilizers
	\item Pharmaceuticals
\end{itemize}

\section{Difference between Domestic Trade and Foreign Trade}
The following are the distinguishing features between Domestic and Foreign Trade:

\begin{enumerate}
	\item \textbf{Mobility of Factors of Production}:
	      There is free movement of labour and capital in Domestic Trade, while the movement of labour and capital is restricted in Foreign Trade.
	\item \textbf{Use of Different Currencies}:
	      Domestic Trade uses the local currency, whereas Foreign Trade uses foreign currencies.
	\item \textbf{Movement of Goods}:
	      It is easy to move goods and services within a country in Domestic Trade, while cross-border movement is more restricted in Foreign Trade.
	\item \textbf{Market Size}:
	      Foreign Trade expands the market for goods and services beyond national boundaries, while Domestic Trade depends on the domestic population, offering a limited market.
	\item \textbf{Transportation Costs}:
	      Transportation costs are usually higher in Foreign Trade due to longer distances and border procedures.
\end{enumerate}

\section{Advantages of Foreign Trade}
The advantages of foreign trade also reflect its importance in the economic development of nations. Key advantages include:

\begin{enumerate}
	\item It enables countries to obtain goods and services that they cannot produce locally.
	\item It widens the market for domestic products by providing access to international markets.
	\item It encourages \textbf{specialisation} and efficient use of resources.
	\item It provides access to advanced technology and innovation from other countries.
	\item It promotes international cooperation and relationships among nations.
	\item It creates employment opportunities in export-oriented industries.
	\item It increases consumer choice by providing a wider variety of goods and services.
\end{enumerate}

\section{Disadvantages of Foreign Trade}
Despite its many advantages, foreign trade also has some drawbacks:

\begin{enumerate}
	\item Over-dependence on imports may weaken local industries and discourage self-reliance.
	\item Trade imbalances (where imports exceed exports) can lead to economic instability.
	\item Fluctuations in international prices may negatively affect exporters and importers.
	\item It can expose domestic industries to unfair competition from stronger foreign companies.
	\item Over-reliance on a few export commodities makes countries vulnerable to global market shocks.
	\item Political conflicts or sanctions can disrupt trade relations between countries.
	\item Excessive foreign trade may undermine cultural values by promoting foreign lifestyles and products.
\end{enumerate}
\section{Challenges of Foreign Trade}
Countries like Malawi face several challenges in conducting foreign trade, including:

\begin{enumerate}
	\item Poor infrastructure such as roads, railways, and ports which increases transportation costs.
	\item Shortage of foreign exchange, making it difficult to pay for imports.
	\item Over-reliance on agricultural exports which are vulnerable to price fluctuations and weather conditions.
	\item Trade barriers such as tariffs and quotas imposed by other countries.
	\item Limited industrial base, making it difficult to diversify exports.
	\item Political instability or policy inconsistencies that discourage investors and trade partners.
\end{enumerate}

\section{Balance of Trade}
\textbf{Balance of Trade (BOT)} is the difference between the value of a country’s exports and the value of its imports within a given period, usually a year.

\[
	\text{Balance of Trade} = \text{Value of Exports} - \text{Value of Imports}
\]

The Balance of Trade can either be favourable or unfavourable depending on whether exports exceed imports or not.

\subsection{Favourable and Unfavourable Balance of Trade}
\begin{itemize}
	\item \textbf{Favourable Balance of Trade}:
	      Occurs when the value of a country’s exports is greater than the value of its imports. This is desirable because it earns the country foreign exchange and promotes economic growth.

	\item \textbf{Unfavourable Balance of Trade}:
	      Occurs when the value of a country’s imports is greater than the value of its exports. This situation is undesirable because it may lead to depletion of foreign reserves and dependency on other nations.
\end{itemize}

\section{Balance of Payments}
The \textbf{Balance of Payments (BOP)} is a record of all monetary transactions made between residents of a country and the rest of the world during a specific period, usually a year.
It includes transactions related to \textbf{exports and imports of goods and services}, as well as financial flows such as \textbf{loans, investments, and remittances}.

Unlike the \textbf{Balance of Trade}, which only considers visible exports and imports of goods, the Balance of Payments is broader and includes both visible and invisible items.

\textbf{Main components of the Balance of Payments:}
\begin{enumerate}
	\item \textbf{Current Account}:
	      Records the export and import of goods (visible trade), services (invisible trade), income flows, and current transfers.

	\item \textbf{Capital Account}:
	      Records capital transfers such as debt forgiveness, transfer of ownership of fixed assets, and migrant transfers.

	\item \textbf{Financial Account}:
	      Records international investments and financial flows such as foreign direct investment (FDI), portfolio investment, and changes in reserves.
\end{enumerate}

A Balance of Payments is said to be:
\begin{itemize}
	\item \textbf{Favourable} when inflows of foreign exchange exceed outflows.
	\item \textbf{Unfavourable} when outflows of foreign exchange exceed inflows.
\end{itemize}

\vspace{0.5cm}

\section{Terms Used in Foreign Trade}
There are several important terms commonly used in foreign trade:

\begin{enumerate}
	\item \textbf{Visible Trade}: Trade in physical goods that can be touched and seen, such as machinery, cars, and agricultural products.
	\item \textbf{Invisible Trade}: Trade in services such as banking, insurance, tourism, transport, and consultancy.
	\item \textbf{Balance of Trade (BOT)}: The difference between the value of exports and imports of \textit{goods} only.
	\item \textbf{Balance of Payments (BOP)}: A record of all financial transactions between a country and the rest of the world, including goods, services, and financial flows.
	\item \textbf{Tariff}: A tax imposed on imported goods to protect local industries or raise government revenue.
	\item \textbf{Quota}: A limit on the quantity of a particular good that can be imported into a country.
	\item \textbf{Exchange Rate}: The value of one country’s currency expressed in terms of another currency.
	      Example: 1 US Dollar = 1,700 Malawi Kwacha.
	\item \textbf{Free Trade}: Trade between countries that is conducted without restrictions such as tariffs or quotas.
	\item \textbf{Protectionism}: The use of trade barriers to protect local industries from foreign competition.
	\item \textbf{Dumping}: The practice of selling goods in a foreign market at prices lower than their cost of production or below the domestic market price.
\end{enumerate}
%\chapter{Data Analytics Journey}
%\section{Learning Python}
%\lipsum[4]

%\section{Building Projects}
%\lipsum[5]

% ---- BACK MATTER ----
\backmatter
%\chapter*{Appendix}
%\lipsum[6]

\nocite{*}
\printbibliography{}
\end{document}

