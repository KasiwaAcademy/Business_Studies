\documentclass[12pt,a4paper, openany]{book}

% =========================
% PACKAGES
% =========================
\usepackage[utf8]{inputenc}
\usepackage[T1]{fontenc}
\usepackage{lipsum}
\usepackage{graphicx}
\usepackage{hyperref}
\usepackage{amsmath, amssymb}
% =========================
% For APA referencing
\usepackage[style=apa,backend=biber]{biblatex}
\DeclareLanguageMapping{english}{english-apa}

% Add your .bib file
\addbibresource{business_studies_form_3.bib}
% =========================
% BEGIN DOCUMENT
% =========================
\begin{document}

% ---- FRONT MATTER ----
\frontmatter
\title{Business Studies \\ form 3}
\author{Sir Brown}
\date{\today}

\maketitle

\chapter*{Dedication}
To my students at Kamuzu Barracks CDSS.

\chapter*{Preface}
This book provides clear and easy-to-understand notes for Business Studies.
The motivation for writing it arose from the observation that many of my students
do not have access to quality Business Studies materials, and that most existing
books often contain content that is even challenging for undergraduate students.
To make high school Business Studies more accessible and understandable,
I created this book.

\tableofcontents

% ---- MAIN MATTER ----
\mainmatter
\chapter{Foreign Trade}

\section{Introduction}
Trade is the exchange of \textbf{goods} and \textbf{services} with the main aim of making \textbf{profits}.
The exchange of goods and services may take place within a country or across countries.
This gives rise to two main types of trade:
\begin{enumerate}
	\item \textbf{Domestic Trade}
	\item \textbf{Foreign Trade}
\end{enumerate}

\section{Domestic Trade}
Domestic Trade is the exchange of goods and services between individuals or organisations within the same country.
It is mostly characterised by the use of the \textbf{local currency}.
This type of trade is also called \textbf{Home Trade} or \textbf{Local Trade}.

\section{Foreign Trade}

\subsection{Meaning of Foreign Trade}
Foreign Trade is the exchange of goods, services, and capital across the boundaries of a country.
It may involve governments, individuals, or organisations in different countries.
For example, Malawi sells tobacco to Japan and America, and imports cars and cellphones from China and India.
Since countries use different currencies, transactions in Foreign Trade are often conducted using common currencies such as the US Dollar (\$) or the South African Rand (R).

\subsection{Basis of Foreign Trade}
Foreign Trade exists because different countries have different resource \textbf{endowments}.
For example:
\begin{itemize}
	\item China produces cars and cellphones that Malawi cannot produce.
	\item Malawi produces tobacco in large quantities, which China cannot.
\end{itemize}

These differences make countries \textbf{specialise} in producing certain commodities, and then trade with others to obtain goods they cannot produce themselves.

\textbf{Types of Foreign Trade:}
\begin{itemize}
	\item \textbf{Bilateral Trade}: Trade between two countries only.
	\item \textbf{Multilateral Trade}: Trade involving more than two countries.
\end{itemize}

\subsection{Divisions of Foreign Trade}
Foreign trade can be divided into three main categories:

\begin{enumerate}
	\item \textbf{Import Trade}: Buying goods and services from other countries.
	      Example: Malawi imports \textit{cars} from Japan and \textit{cellphones} from China.

	\item \textbf{Export Trade}: Selling domestic goods and services to other countries.
	      Example: Malawi exports \textit{tobacco, tea, and sugar}.

	\item \textbf{Entrepot Trade}: Re-exporting goods.
	      Example: Singapore imports crude oil, refines it, and then re-exports petroleum products.
\end{enumerate}

\subsection{Major Imports and Exports of Malawi}
\textbf{Major Exports:}
\begin{itemize}
	\item Tobacco
	\item Tea
	\item Sugar
	\item Coffee
	\item Groundnuts
\end{itemize}

\textbf{Major Imports:}
\begin{itemize}
	\item Motor vehicles and spare parts
	\item Petroleum and petroleum products
	\item Machinery and equipment
	\item Fertilisers
	\item Pharmaceuticals
\end{itemize}

\section{Difference between Domestic and Foreign Trade}
\begin{enumerate}
	\item \textbf{Mobility of Factors}: Labour and capital move freely in Domestic Trade, but are restricted in Foreign Trade.
	\item \textbf{Currency}: Domestic Trade uses local currency; Foreign Trade uses foreign currencies.
	\item \textbf{Movement of Goods}: Goods move easily within a country, but cross-border trade faces restrictions and customs procedures.
	\item \textbf{Market Size}: Domestic Trade is limited to the local population; Foreign Trade widens the market internationally.
	\item \textbf{Transport Costs}: Higher in Foreign Trade due to long distances and border charges.
\end{enumerate}

\section{Advantages of Foreign Trade}
\begin{enumerate}
	\item Provides goods and services not produced locally.
	\item Expands markets for Malawian products.
	\item Encourages \textbf{specialisation} and efficient resource use.
	\item Gives access to advanced technology.
	\item Strengthens international cooperation.
	\item Creates employment in export industries.
	\item Increases consumer choice.
\end{enumerate}

\section{Disadvantages of Foreign Trade}
\begin{enumerate}
	\item Over-dependence on imports weakens local industries.
	\item Trade imbalances can harm the economy.
	\item Price fluctuations on world markets affect Malawi’s exports.
	\item Local industries face stiff competition from foreign firms.
	\item Reliance on a few crops (like tobacco) is risky.
	\item Political conflicts or sanctions can disrupt trade.
	\item Foreign goods may weaken local culture.
\end{enumerate}

\section{Challenges of Foreign Trade in Malawi}
\begin{enumerate}
	\item Poor infrastructure (roads, railways, and ports).
	\item Shortage of foreign exchange.
	\item Over-reliance on agriculture.
	\item Trade barriers (tariffs and quotas) from other countries.
	\item Limited industries and exports.
	\item Political instability and policy changes.
\end{enumerate}

\section{Balance of Trade}
\textbf{Balance of Trade (BOT)} is the difference between the value of exports and imports within a year.

\[
	\text{Balance of Trade} = \text{Exports} - \text{Imports}
\]

\begin{itemize}
	\item \textbf{Favourable BOT}: Exports > Imports (earns foreign exchange).
	\item \textbf{Unfavourable BOT}: Imports > Exports (loss of foreign exchange).
\end{itemize}

\section{Balance of Payments}
\textbf{Balance of Payments (BOP)} is a record of all transactions between Malawi and the rest of the world in a year.
It includes both goods and services, as well as financial flows like loans and remittances.

\textbf{Components:}
\begin{enumerate}
	\item \textbf{Current Account}: Exports, imports, services, and transfers.
	\item \textbf{Capital Account}: Transfers of assets, debt forgiveness.
	\item \textbf{Financial Account}: Investments, loans, and changes in reserves.
\end{enumerate}

\subsection{Favourable Balance of Payment (BoP)}
\begin{itemize}
	\item Occurs when a country's inflow of \textbf{foreign exchange} is greater than its outflows.
	\item The nation earns more from exports of goods and services than it spends on imports and other outflows.
	\item It usually mean a country has a \textbf{BoP} surplus.
	\item \textbf{Favourable BOP}: Inflows of foreign exchange > outflows.
\end{itemize}

\subsection{Unfavourable Balance of Payment (Bop)}
\begin{itemize}
	\item Occurs when a country's outflows of foreign exchange is greater than its inflows.
	\item The nation spends more on imports than it earns on exports and other foreign transactions.
	\item It usually means that the country has a \textbf{BoP} surplus.
	\item \textbf{Unfavourable BOP}: Outflows > inflows.
\end{itemize}

\section{Terms Used in Foreign Trade}
\begin{enumerate}
	\item \textbf{Visible Trade}: Trade in goods (e.g. cars, crops).
	\item \textbf{Invisible Trade}: Trade in services (e.g. banking, tourism).
	\item \textbf{Tariff}: A tax on imports. The custom duty that is paid on goods
	      entering the country. Because of the tarrifs, the prices of goods and services
	      on the local market rises.
	\item \textbf{Quota}: A limit on the quantity of imports. A country gives a limit
	      on quantity or value of the import of particulr commodities or goods of a particular
	      country.
	\item \textbf{Exchange Rate}: Value of one currency compared to another.
	      Example: 1 USD = MK 1,700.
	\item \textbf{Free Trade} or \textbf{Trade Liberalization}: Trade without restrictions. The
	      free flow of goods and services into or out of the country.
	\item \textbf{Protectionism}: Use of trade barriers to protect local industries.
	      A country puts restrictions or barriers on imports or sometimes export aimed at
	      protecting local industries / businesses.
	\item \textbf{Embargo}: A total ban on the importation of a particular commodity or goods from
	      a particular country.
	\item \textbf{Dumping}: Selling goods abroad at very low prices. The export of goods to countries where
	      they are sold at a lower price than they fetch on the firm's home market. Dumping is usually done to kill
	      competing industries of the importing country.
	\item \textbf{Devaluation}: This is when the currency of the particular country falls in value
	      against other internationally recognised currencies such as the US Dollar (\$).
	\item \textbf{Trade Agreements}: A situation where countries negotiate certain trade incentives in order to make
	      trade free or partially free. These are also called \textbf{Trade Protocols}.
\end{enumerate}

\section*{Summary / Revision Points}
\begin{itemize}
	\item Trade is divided into \textbf{Domestic} (within Malawi) and \textbf{Foreign} (between Malawi and other countries).
	\item Foreign Trade exists because countries have different resources and specialise in certain goods.
	\item Malawi mainly \textbf{exports}: tobacco, tea, sugar, coffee, and groundnuts.
	\item Malawi mainly \textbf{imports}: vehicles, fuel, fertilisers, machinery, and medicines.
	\item Foreign Trade can be divided into \textbf{imports, exports, and entrepot trade}.
	\item Advantages include access to goods, wider markets, technology, and jobs.
	\item Disadvantages include over-dependence, trade imbalances, and stiff competition.
	\item Malawi faces challenges such as poor infrastructure, lack of forex, and reliance on agriculture.
	\item \textbf{BOT} = Exports – Imports. Can be favourable or unfavourable.
	\item \textbf{BOP} is broader than BOT and includes all international transactions.
	\item Key terms: tariffs, quotas, exchange rate, dumping, free trade, and protectionism.
\end{itemize}

%\chapter{Data Analytics Journey}
%\section{Learning Python}
%\lipsum[4]

%\section{Building Projects}
%\lipsum[5]

% ---- BACK MATTER ----
\backmatter
%\chapter*{Appendix}
%\lipsum[6]

\nocite{*}
\printbibliography{}
\end{document}

