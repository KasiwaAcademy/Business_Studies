\documentclass[12pt,a4paper, openany]{book}

% =========================
% PACKAGES
% =========================
\usepackage[utf8]{inputenc}
\usepackage[T1]{fontenc}
\usepackage{lipsum}
\usepackage{graphicx}
\usepackage{hyperref}
% =========================
% For APA referencing
\usepackage[style=apa,backend=biber]{biblatex}
\DeclareLanguageMapping{english}{english-apa}

% Add your .bib file
\addbibresource{business_studies_form_3.bib}
% =========================
% BEGIN DOCUMENT
% =========================
\begin{document}

% ---- FRONT MATTER ----
\frontmatter
\title{Business Studies \\ form 3}
\author{Sir Brown}
\date{\today}

\maketitle

\chapter*{Dedication}
To my students at Kamuzu Barracks CDSS.

\chapter*{Preface}
This book provides clear and easy-to-understand notes for Business Studies.
The motivation for writing it arose from the observation that many of my students
do not have access to quality Business Studies materials, and that most existing
books often contain content that is even challenging for undergraduate students.
To make high school Business Studies more accessible and understandable,
I created this book.

\tableofcontents

% ---- MAIN MATTER ----
\mainmatter
\chapter{Foreign Trade}

\section{Introduction}
Trade is defined as the exchange of \textbf{goods} and \textbf{services} with the main aim of making \textbf{profits}.
The exchange of goods and services may take place within a country or across countries.
This gives rise to two types of trade, namely:
\begin{enumerate}
	\item \textbf{Domestic Trade}
	\item \textbf{Foreign Trade}
\end{enumerate}

\section{Domestic Trade}
Domestic Trade is the exchange of goods and services between individuals or organisations located within the same country.
It is mostly characterised by the use of the \textbf{local currency}.
This type of trade is also called \textbf{Home Trade} or \textbf{Local Trade}.

\section{Foreign Trade}

\subsection{Meaning of Foreign Trade}
Foreign Trade is the exchange of capital, goods, and services across the boundaries of a country.
It may occur between governments, individuals, or organisations in different countries.
For example, Malawi sells its tobacco to Japan and America, and imports cars and cellphones from China and India.
Since Foreign Trade takes place between countries that use different currencies, transactions are usually conducted using a common currency such as the US Dollar (\$) or South African Rand (R).

\subsection{Basis of Foreign Trade}
Different countries have different resource \textbf{endowments}.
For example, China can produce cars and cellphones, which Malawi cannot, while Malawi produces tobacco, which China cannot produce in sufficient quantities.
These differences in resources and capabilities give rise to Foreign Trade by allowing countries to buy goods and services they cannot produce themselves, and sell surplus production to other countries.
No country can produce all the goods and services required by its citizens.
Countries \textbf{specialise} in producing certain commodities and rely on Foreign Trade to acquire other commodities produced by other countries.

\vspace{0.5cm}

It is important to note that Foreign Trade may occur between two countries, which is called \textbf{Bilateral Trade}.
It may also involve more than two countries, in which case it is referred to as \textbf{Multilateral Trade}.

\section{Difference between Domestic Trade and Foreign Trade}
The following are the distinguishing features between Domestic and Foreign Trade:

\begin{enumerate}
	\item \textbf{Mobility of Factors of Production}:
	      There is free movement of labour and capital in Domestic Trade, while the movement of labour and capital is restricted in Foreign Trade.
	\item \textbf{Use of Different Currencies}:
	      Domestic Trade uses the local currency, whereas Foreign Trade uses foreign currencies.
	\item \textbf{Movement of Goods}:
	      It is easy to move goods and services within a country in Domestic Trade, while cross-border movement is more restricted in Foreign Trade.
	\item \textbf{Market Size}:
	      Foreign Trade expands the market for goods and services beyond national boundaries, while Domestic Trade depends on the domestic population, offering a limited market.
	\item \textbf{Transportation Costs}:
	      Transportation costs are usually higher in Foreign Trade due to longer distances and border procedures.
\end{enumerate}

%\chapter{Data Analytics Journey}
%\section{Learning Python}
%\lipsum[4]

%\section{Building Projects}
%\lipsum[5]

% ---- BACK MATTER ----
\backmatter
%\chapter*{Appendix}
%\lipsum[6]

\nocite{*}
\printbibliography{}
\end{document}

