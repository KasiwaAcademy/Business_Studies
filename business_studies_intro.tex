\documentclass[12pt,a4paper, openany]{book}
\usepackage{geometry}
\usepackage{titlesec}
\usepackage{enumitem}
\usepackage{amsmath}
\usepackage{multicol}
\usepackage{hyperref}

\geometry{margin=1in}
\setlist{nosep}

\title{Introduction to Business Studies}
\author{Senior Secondary School Notes}
\date{}

\begin{document}
\maketitle

\chapter*{Introduction to Business Studies}
\addcontentsline{toc}{chapter}{Introduction to Business Studies}

\section{Definition of a Business}
A \textbf{business} is any activity carried out by individuals or organisations with the aim of:
\begin{itemize}
	\item Providing goods and services to satisfy human needs and wants.
	\item Making a profit in return.
\end{itemize}

\textbf{Examples:} shops, farms, transport services, barbershops, and banks.

\section{Definition of Business Studies}
\textbf{Business Studies} is the study of how businesses are started, organised, operated, and managed.
It explains how goods and services are exchanged, how resources are used, and how businesses contribute to the community and the economy.

\section{Importance of Business Studies}
Studying Business helps learners to:
\begin{enumerate}
	\item Gain knowledge and skills for running and managing businesses.
	\item Understand how goods and services are produced and distributed.
	\item Develop entrepreneurial skills and creativity.
	\item Learn how to make better financial decisions in daily life.
	\item Prepare for careers in business, accounting, marketing, or management.
	\item Contribute to the economic development of the country.
\end{enumerate}

\section{Contributions of Business to the National Economy}
Businesses help the nation in many ways:
\begin{enumerate}
	\item \textbf{Employment creation} - providing jobs for people.
	\item \textbf{Government revenue} - paying taxes that fund schools, hospitals, and infrastructure.
	\item \textbf{Production of goods and services} - meeting the needs of society.
	\item \textbf{Improving standards of living} - making goods and services available.
	\item \textbf{Foreign exchange earnings} - exporting products like tobacco, tea, and sugar.
	\item \textbf{Development of infrastructure} - businesses encourage building of roads, electricity, and communication networks.
\end{enumerate}

\section{Starting up a Business}
When starting a business, an entrepreneur must consider:
\begin{enumerate}
	\item \textbf{Business idea} - what goods or services will be provided.
	\item \textbf{Capital} - money needed to start operations.
	\item \textbf{Location} - where the business will operate.
	\item \textbf{Market} - who will buy the product or service.
	\item \textbf{Legal requirements} - licenses, permits, and registration.
	\item \textbf{Labour} - workers with the right skills.
	\item \textbf{Management} - organising and controlling the business effectively.
\end{enumerate}

\section{Sources of Money to Start a Business}
An entrepreneur can raise money from:
\begin{itemize}
	\item Personal savings.
	\item Borrowing from friends and relatives.
	\item Bank loans and microfinance institutions.
	\item Government grants or youth development funds.
	\item Cooperative societies or village banks.
	\item Retained profits (for existing businesses).
\end{itemize}

\section{Summary / Revision Points}
\begin{itemize}
	\item A business is an activity providing goods and services, usually for profit.
	\item Business Studies teaches how businesses operate, manage resources, and support society.
	\item Importance: develops entrepreneurship, creates knowledge, and prepares students for future careers.
	\item Businesses contribute to the economy through jobs, taxes, exports, and better living standards.
	\item To start a business: consider the idea, capital, location, market, labour, and legal requirements.
	\item Sources of capital: savings, loans, relatives, grants, and profits.
\end{itemize}

\section*{Practice Questions}
\addcontentsline{toc}{section}{Practice Questions}
\begin{enumerate}
	\item Define the term \textbf{business}. Give two examples from Malawi.
	\item Explain three reasons why it is important to study Business Studies.
	\item List four ways in which businesses contribute to the Malawian economy.
	\item State three factors an entrepreneur should consider before starting a business.
	\item Identify three possible sources of money to start a small business.
\end{enumerate}

\end{document}

